% COPYRIGHT {{{

    % Copyright (C) 2016 Manuel Castillo López

    % This program is free software: you can redistribute it and/or modify
    % it under the terms of the GNU General Public License as published by
    % the Free Software Foundation, either version 3 of the License, or
    % (at your option) any later version.

    % This program is distributed in the hope that it will be useful,
    % but WITHOUT ANY WARRANTY; without even the implied warranty of
    % MERCHANTABILITY or FITNESS FOR A PARTICULAR PURPOSE.  See the
    % GNU General Public License for more details.

    % You should have received a copy of the GNU General Public License
    % along with this program.  If not, see <http://www.gnu.org/licenses/>.

% }}}

\documentclass[../root.tex]{subfiles}
\begin{document}


\chapter{Abstract}
Theoretical physics is all about casting our concepts about the real world into rigorous mathematical form. But, theoretical physics doesn't do that for its own sake. It does so, in order to fully explore the implications of what our concepts about the real world are. To certain extent, the spirit of theoretical physics can be casted into the words of Wittengstein that said: What we cannot speak about clearly then we must pass over in silence. Indeed, if we have concepts about the real world and it's not possible to cast them into precise mathematical language, then usually that is an indicator that some aspects of these concepts have not been well understood. But then mathematics is just that, a language. And, if we want to extract physical conclusions from this formulation we must interpret the language. But again citing Wittengstein, the theorems of mathematics all say the same: namely nothing. That doesn't mean that mathematics is useless. He just refers to the fact that if we have a theorem of the type: A if and only if B. A and B being propositions, then obviously B says nothing else than A does. And A says nothing else than B does. It is what is called in mathematics a tautology. However, psychologically, for our understanding of A, it may be very useful to have a reformulation of A in terms of B. Therefore, the objective of the course is to provide fundamental mathematical language to build any concept addressed by science.


\end{document}
